\documentclass[11pt,french]{article}
\usepackage[a5paper,landscape]{geometry}
\geometry{verbose,tmargin=15mm,bmargin=14mm,lmargin=12mm,rmargin=12mm}
\usepackage{courier}
\usepackage{libertine}
\usepackage[T1]{fontenc}
\usepackage[utf8]{inputenc}
\usepackage[francais]{babel}
\usepackage{lmodern}
\usepackage{textcomp}
\usepackage{listingsutf8}    % listingsutf8 to allow UTF8 accentuated characters !
\usepackage{graphicx}
\usepackage{lastpage}
\usepackage{tabls}
\usepackage{color}
\usepackage{colortbl}
\usepackage{array}
\usepackage{float}
\usepackage{makeidx}
\usepackage{amsmath,amssymb,amsfonts}
\usepackage{babel}
\usepackage{alltt}
\usepackage[table]{xcolor}
\usepackage{layouts}
\usepackage{pifont}
\usepackage[sc]{mathpazo}
\usepackage{marvosym}
\usepackage{tikz}
\usepackage{fancyhdr}
\usepackage{tabularx}
\usepackage{wasysym}
\usepackage{manfnt}
\usepackage{algorithmic}
\usepackage{algorithm}
\pagestyle{fancy}
\setlength{\parskip}{\smallskipamount}
\setlength{\parindent}{0pt}
\usepackage{fancybox}
\usepackage{calc}
%\usepackage{setspace}

%%%%%%%%%%%%%%%%%%%%%%%%%%%%%% User specified LaTeX commands.
\usepackage{graphics}
\usepackage{enumerate}
\usepackage{enumitem}
\usepackage{hhline}
\usepackage{multicol}
\usepackage{upquote}
\usepackage{stmaryrd}  % rrbracket et llbracket
\usepackage[tikz]{bclogo}
\usepackage[
  hypertexnames=false, % for correct links (duplicate-error solution)
  setpagesize=false,   % necessary in order to not change text-/paperformat for the document
  pdfborder={0 0 0},   % removes border around links
  %pdfpagemode=FullScreen,% open pdf in full screen mode
  pdfstartview=Fit% fit page to pdf viewer
]{hyperref}
\hypersetup{colorlinks,
  citecolor = red,
  filecolor = black,
  linkcolor =,
  urlcolor  = blue}

\hyphenpenalty=100000

%%%%%%%%%%%%%%%%%%  Couleurs %%%%%%%%%%%%%%%%%%%%%%%%%%%
\definecolor{mauve}{rgb}{0.5,0,0.7}
\definecolor{carmin}{rgb}{0.7,0,0}
\definecolor{bleu}{rgb}{0,0,0.7}
\definecolor{marron}{rgb}{0.6,0.35,0}
\definecolor{vert}{rgb}{0,0.5,0}
\definecolor{gris}{rgb}{0.9,0.9,0.9}
\definecolor{no}{rgb}{1,0.9,1}
%%%%%%%%%%%%%%%%%%%%%%%%%%%%%%%%%%%%%%%%%%%%%%%%%%%%%%%%%%

\definecolor{White}        {rgb}{1.0,1.0,1.0}

\definecolor{VeryDarkBlue} {RGB}{0,0,153}
\definecolor{DarkBlue}     {rgb}{0.0,0.0,0.6}
\definecolor{Blue}         {rgb}{0.0,0.0,1.0}
\definecolor{MidBlue}      {rgb}{0.6,0.6,1.0}
\definecolor{LightBlue}    {rgb}{0.8,0.8,1.0}
\definecolor{VeryLightBlue}{rgb}{0.9,0.9,1.0}

\definecolor{Gray}{rgb}{0.7,0.7,0.7}
\definecolor{LightGray}{rgb}{0.94,0.94,0.94}

\definecolor{DarkGreen}{rgb}{0,0.6,0}
\definecolor{MidGreen}{rgb}{0.6,1,0.6}
\definecolor{LightGreen}{rgb}{0.88,1,0.88}
\definecolor{VeryLightGreen}{rgb}{0.9,1,0.9}

\definecolor{Yellow}{rgb}{1,1,0.4}
\definecolor{MidYellow}{rgb}{1,1,0.5}
\definecolor{LightYellow}{rgb}{1,1,0.6}
\definecolor{VeryLightYellow}{rgb}{1,1,0.9}

\definecolor{DarkRed}{rgb}{0.7,0.,0.}
\definecolor{Red}{rgb}{0.8,0.,0.}
\definecolor{LightRed}{rgb}{1,0.8,0.8}
\definecolor{VeryLightRed}{rgb}{1,0.9,0.9}

\definecolor{Mauve}{rgb}{0.7,0.,0.7}

\definecolor{Magenta}{rgb}{1,0.,1}
\definecolor{LightMagenta}{rgb}{1,0.5,1}

\definecolor{Chocolate}{rgb}{0.54,0.2,.004}
\definecolor{DarkChocolate}{RGB}{103,38,.00}

\definecolor{DarkOrange}{rgb}{0.65,0.35,0.}
\definecolor{Orange}{rgb}{.9,0.4,0.}
\definecolor{LightOrange}{rgb}{1,0.8,0.5}

\definecolor{Brown}{rgb}{0.4,0.4,0.}

\definecolor{LightCyan}{rgb}{0.5,1.,1.}

\definecolor{PBG} {rgb}{1.0,1.0,0.6}
\definecolor{PTXT}{rgb}{0.2,0.2,0.0}
\definecolor{IBG} {rgb}{0.8,0.8,1.0}
\definecolor{ITXT}{rgb}{0.2,0.2,0.0}
\definecolor{OBG} {rgb}{0.8,1.0,0.8}
\definecolor{OTXT}{rgb}{0.0,0.4,0.0}

\newcommand{\CommentVeryLightBlue}[1]{{\textcolor{VeryLightBlue}{#1}}}
\newcommand{\CommentWhite}[1]{{\textcolor{White}{#1}}}

\newcommand{\VeryDarkBlue}[1]{\textcolor{DarkBlue}{#1}}
\newcommand{\DarkBlue}[1]{\textcolor{DarkBlue}{#1}}
\newcommand{\Blue}[1]{\textcolor{Blue}{#1}}
\newcommand{\LightBlue}[1]{\textcolor{LightBlue}{#1}}
\newcommand{\VeryLightBlue}[1]{\textcolor{VeryLightBlue}{#1}}

\newcommand{\DarkGreen}[1]{\textcolor{DarkGreen}{#1}}
\newcommand{\LightGreen}[1]{\textcolor{LightGreen}{#1}}
\newcommand{\VeryLightGreen}[1]{\textcolor{VeryLightGreen}{#1}}
\newcommand{\MidGreen}[1]{\textcolor{MidGreen}{#1}}

\newcommand{\DarkRed}[1]{\textcolor{DarkRed}{#1}}
\newcommand{\Red}[1]{\textcolor{red}{#1}}
\newcommand{\LightRed}[1]{\textcolor{LightRed}{#1}}
\newcommand{\VeryLightRed}[1]{\textcolor{VeryLightRed}{#1}}

\newcommand{\Gray}[1]{\textcolor{gray}{#1}}

\newcommand{\Black}[1]{\textcolor{black}{#1}}

\newcommand{\White}[1]{\textcolor{white}{#1}}

\newcommand{\Mauve}[1]{\textcolor{Mauve}{#1}}

\newcommand{\LightMagenta}[1]{\textcolor{LightMagenta}{#1}}
\newcommand{\Magenta}[1]{\textcolor{Magenta}{#1}}

\newcommand{\DarkOrange}[1]{\textcolor{DarkOrange}{#1}}
\newcommand{\Orange}[1]{\textcolor{Orange}{#1}}
\newcommand{\LightOrange}[1]{\textcolor{LightOrange}{#1}}

\newcommand{\DeepPurple}[1]{{\textcolor[rgb]{0.3,0.,0.6}{#1}}}

\newcommand{\Brown}[1]{{\textcolor{Brown}{#1}}}
\newcommand{\Chocolate}[1]{\textcolor{Chocolate}{#1}}
\newcommand{\DarkChocolate}[1]{\textcolor{DarkChocolate}{#1}}

\definecolor{DBluePy}{RGB}{ 28, 78, 99}
\definecolor{BluePy} {RGB}{ 60,110,131}
\definecolor{MBluePy}{RGB}{200,210,240}
\definecolor{LBluePy}{RGB}{220,230,240}
\definecolor{DOranPy}{RGB}{248,194,  3}
\definecolor{OranPy} {RGB}{255,220, 70}
\definecolor{GreenPy}{RGB}{237,255,204}

\lstdefinestyle{MyPython}    
{%configuration de listings %
 language=Python,                                       %
 extendedchars=true,                                    %
 %inputencoding = utf8,                           %
 backgroundcolor=\color{LBluePy},                    %
 basicstyle=\color{black}\fontsize{9}{9}\selectfont\ttfamily\bfseries,
 numberstyle=\color{BluePy}\fontsize{8}{8}\selectfont\ttfamily,   
 %identifierstyle=\color{Blue}\ttfamily,                 %
 keywordstyle=\color{Orange}\ttfamily\bfseries,         %
 stringstyle=\color{DarkGreen}\ttfamily\bfseries,       %
 commentstyle=\color{red}\ttfamily,                     %
 emph={print,str,@property},emphstyle=\color{Mauve}\ttfamily\bfseries,
 %procnamekeys={def,class},procnamestyle=\color{blue}\ttfamily\bfseries,
 tabsize=2,                                             %
 frame=none,                                          %
% frameround=ffff,                                      %
 extendedchars=true,                                    %
 literate={à}{{\`a}}1 {ç}{{\c{c}}}1 {Ç}{{\c{C}}}1 {ê}{{\^e}}1 {é}{{\'e}}1 {è}{{\`e}}1,
 language=Python,                                 % the language of the code
 deletekeywords={exec, super, None},                               % if you want to delete keywords from the given language
 morekeywords={[2]super},keywordstyle={[2]\bfseries\Mauve},
 morekeywords={[3]as, None},keywordstyle={[3]\bfseries\Orange},
 showspaces=false,                                      %
 showstringspaces=false,                                %
 showtabs=false,                                        %
 numbers=left,                                          %
 stepnumber=1,                                          %
 breaklines=true,                                       %
 linewidth=\linewidth,                                  %  
% showlines=true,                                       %
% captionpos=b,                                         %
 xleftmargin=1cm,                                       %
 xrightmargin=0.3cm,                                      %
 morekeywords={},                                        %
 texcl=true,
 rulecolor=\color{BluePy},
 morestring=[s][\color{DarkGreen}]{'''}{'''},
 }

 % python listing configuration

\lhead{\footnotesize APP -- {\em Machine Learning}}
\chead{\footnotesize }
\rhead{\footnotesize Version 2.0 -- 12 avril 2022}
\lfoot{\fontsize{9}{9}\selectfont\textbf{Arts \& Métiers ParisTech}~-~\textit{Bordeaux}}
\cfoot{\fontsize{9}{9}\selectfont\thepage~/~\pageref{LastPage}}
\rfoot{\fontsize{9}{9}\selectfont{UEF 2A -- Mathématiques et Informatique}}
\renewcommand{\footrulewidth}{0.3pt}
\renewcommand{\headrulewidth}{0.3pt}

\AtBeginDocument{
  \renewcommand{\labelitemi}{$\rhd$}
  \renewcommand{\labelitemii}{$\displaystyle \diamond$}
  \renewcommand{\labelitemiii}{$\displaystyle \circ$}
  \renewcommand{\labelitemiv}{--}
  \renewcommand{\tablename}{T\textsc{ableau}~\no}
}

\newcommand{\bsh}{\textbackslash}
\newcommand{\bshbsh}{\textbackslash\textbackslash}
\newcommand{\chevrons}{\ttfamily>\hspace*{-0.25mm}>\hspace*{-0.25mm}>\hspace*{0.28mm}\xspace}
\newcommand{\M}[1]{\Mauve{#1}}
\newcommand{\DG}[1]{\DarkGreen{#1}}
\newcommand{\DR}[1]{\DarkRed{#1}}
\newcommand{\B}[1]{\Blue{#1}}
\newcommand{\BPy}[1]{\BluePy{#1}}
\newcommand{\VDB}[1]{\VeryDarkBlue{#1}}
\newcommand{\DO}[1]{\DarkOrange{#1}}
\newcommand{\Or}[1]{\Orange{#1}}
\newcommand{\Choco}[1]{\Chocolate{#1}}
\newcommand*{\truc}{\Gray{\textbullet}}

\newcommand{\bif}[1]{{\ttfamily \M{#1}}}	% Python built in function
\newcommand{\typ}[1]{{\ttfamily \M{#1}}}  % Python built in type
\newcommand{\key}[1]{{\ttfamily \Or{#1}}}
\newcommand{\str}[1]{{\ttfamily \DG{#1}}}
\newcommand{\com}[1]{{\ttfamily \DR{#1}}}
\newcommand{\out}[1]{{\ttfamily \B{#1}}}
\newcommand{\command}[1]{{\ttfamily \Choco{#1}}}
\newcommand{\code}[1]{{\ttfamily \Choco{#1}}}
\newcommand{\file}[1]{{\ttfamily \VDB{#1}}}

\newcommand{\bifBF}[1]{\textbf{\bif{#1}}}	% Python built in function
\newcommand{\typBF}[1]{\textbf{\typ{#1}}}  % Python built in type
\newcommand{\keyBF}[1]{\textbf{\key{#1}}}
\newcommand{\strBF}[1]{\textbf{\str{#1}}}
\newcommand{\comBF}[1]{\textbf{\com{#1}}}
\newcommand{\outBF}[1]{\textbf{\out{#1}}}
\newcommand{\commandBF}[1]{\textbf{\command{#1}}}
\newcommand{\codeBF}[1]{\textbf{\code{#1}}}
\newcommand{\fileBF}[1]{\textbf{\file{#1}}}

%%%%%%%%%%%%%%%%%%%%%%%%% diapo 1 %%%%%%%%%%%%%%%%%%%%%%%%%%%%%%%%%%%%%%%%%%%%%%%%%%%%%%%%%%%%%%%%%%%
\begin{document}

\setlist[itemize]{leftmargin=5mm,itemsep=0mm}
\setlist[enumerate]{leftmargin=5mm,itemsep=0mm}

\renewcommand{\ttdefault}[0]{lmtt}
\newcommand{\boldtt}[1]{{\ttfamily\bfseries #1}}

\newcommand{\QtLogo}[0]{\includegraphics[width=17pt]{Images/QtLogo.projet}}
\newcommand{\PyQt}[0]{\boldtt{PyQt5}}
\newcommand{\Qt}[0]{\boldtt{Qt}}

\begin{center}
\setlength{\fboxsep}{5mm}
\setlength{\fboxrule}{0.2mm}
\fbox{\Large Unité d'enseignement {\bf\emph{Mathématiques et informatique}}}

\medskip
\fcolorbox{Gray}{LightGray}{
  \begin{tabular}{c}
    {\LARGE Apprentissage Par Projet}\\[3mm]
    {\Mauve{\LARGE {Entraîner un réseau de neurones à classifier des images}}}
  \end{tabular}
  }
\end{center}

\section*{Acquis d'apprentissage visés}

À la fin de cet apprentissage par projet de 3 séances de 3 heures, chaque étudiant saura :
\begin{itemize}
\item[$\rhd$] Programmer en langage Python un réseau de neurones (dense ou convolutif) dédié à la classification d'images
  en utilisant les modules \boldtt{tensorflow2} et \boldtt{keras}.
\item[$\rhd$] Constituer une banque de données d'entraînement et de test.
\item[$\rhd$] Conduire l'entraînement supervisé d'un réseau de neurones en utilisant une banque de données spécifique.
\item[$\rhd$] Exploiter en situation opérationnelle un réseau de neurones entraîné.
\end{itemize}

%%%%%%%%%%%%%%%%%%%% diapo 2 %%%%%%%%%%%%%%%%%%%%%%%%%%%%%%%%%%%%%%%%%%%%%%%%%%%%%%%%%%%
\newpage
\vspace*{-16mm}
\section*{\Large\ding{220} Planification suggérée des 3 séances d'APP}
\vspace*{-3mm}
Au cours des 3 séances, les étudiants travaillent en autonomie par équipe de 3 ou 4
avec leurs ordinateurs portables, dans un \textbf{Environnement Virtuel Python}
(EVP) \DarkRed{\boldtt{minfo\_ml}} spécialement créé pour cet APP.
\vspace*{-4mm}
\begin {bclogo}[noborder=true, couleur=green!10, couleurBarre=DarkGreen, logo=\large\ding{220}]{}
  \vspace*{-4mm}\boldtt{Travail préliminaire} : à faire avant les 3 séances avec un accès Internet haut-débit\\[-5mm]
  \begin{itemize}
  \setlength{\parsep}{-0.2mm}\setlength{\itemsep}{-0.4mm}    
    \item Télécharger les documents de travail depuis l'ENT SAVOIR.
    \item Créer l'EVP \DarkRed{\boldtt{minfo\_ml}} puis installer les modules Python nécessaires au {\em Machine Learning} (détails page suivante).
  \end{itemize} 
\end{bclogo}  
%
\vspace*{-8mm}
\begin {bclogo}[noborder=true, couleurBarre=Chocolate, logo=\large\ding{220}]{}
  \vspace*{-4mm}\boldtt{Séance 1 : auto-formation}\\[-5mm]
  \begin{itemize}
    \item travail personnel avec les trois {\em notebooks} \file{ML1\_MNIST.ipynb}, \file{ML2\_DNN.ipynb}, et \file{ML3\_DNN\_suite.ipynb} pour apprendre à charger les images MNIST, construire un réseau de neurones dense et l'entraîner à reconnaître les images du MNIST.
    \item travail personnel avec le {\em notebook} \file{ML4\_CNN.ipynb} pour apprendre à construire un réseau de neurones convolutif et à l'entraîner à reconnaître les images du MNIST.
  \end{itemize}
\end{bclogo}  
%
\vspace*{-8mm}
\begin {bclogo}[noborder=true, couleurBarre=Chocolate, logo=\large\ding{220}]{}
  \vspace*{-4mm}\boldtt{Séance 2 : fin auto-formation, début projet}\\[-5mm]
  \begin{itemize}
    \item suite et fin du travail avec les {\em notebooks}.
    \item \textbf{Projet} : Recherche d'une banque de données (images ou autre) propre à l'équipe, entraînement supervisé d'un réseau de neurones à classifier les données choisies, exploitation en situation opérationnelle du réseau de neurones entraîné.
  \end{itemize}
\end{bclogo}  
%
\vspace*{-8mm}
\begin {bclogo}[noborder=true, couleurBarre=Chocolate, logo=\large\ding{220}]{}
  \vspace*{-4mm}\boldtt{Séance 3 : fin du projet}\\[-5mm]
  \begin{itemize}
    \item suite et fin du projet. Préparation du rendu.
    \item Dernière heure : Évaluation finale, rendu du projet par équipe.
  \end{itemize}
\end{bclogo}  

%%%%%%%%%%%%%%%%%%%%%%  diapo 3 %%%%%%%%%%%%%%%%%%%%%%%%%%%%%%%%%%%%%%%%%%%%%%%%%%%%%%%%%
\newpage
\vspace*{-16mm}
\section*{\ding{220} Travail péliminaire -- Télécharger les documents, configurer l'EVP \DarkRed{\boldtt{minfo\_ml}}, s'auto-former aux réseaux de neurones}

\vspace*{-1mm}Télécharge l'archive \file{APP-ML$\cdots$.zip} depuis la plate-forme SAVOIR et \Chocolate{\textbf{extrait}} le dossier \file{APP-ML} sur ton ordinateur portable. Une fois extrait, le contenu du dossier \file{APP-ML} est le suivant :
\vspace*{-3mm}
\begin {bclogo}[noborder=true, couleurBarre=Chocolate, logo=\bcinfo]{}
  \vspace*{-5mm}%
  \begin{description}
 \setlength{\parsep}{-0.2mm}\setlength{\itemsep}{-0.2mm}    
 \item[Dossier \file{Cours}] : contient le fichier PDF de l'amphi "Comprendre et utiliser le {\em Machine Learning}".
 \item[Dossier \file{Notebook}] : contient les {\em notebooks} d'auto-formation \file{ML$\cdots$.ipynb}.
 \item[Fichier \file{Consignes.pdf}] : le présent document.
 \item[Fichier \file{requirements.txt}] : liste les modules Python nécessaires au travail de l'APP.
\end{description}
\end{bclogo}  
%
\vspace*{-8mm}
\begin {bclogo}[noborder=true, couleurBarre=Chocolate, logo=\bctrombone]{}
\vspace*{-4mm}\textbf{Créer et configurer l'Environnement Virtuel Python \DarkRed{\boldtt{minfo\_ml}}}\\[-5mm]
  \begin{itemize}
    \item Ouvre une console "Anaconda Prompt" (Windows) ou un terminal (Mac ou Linux).
    \item Créé l'EVP  \DarkRed{\boldtt{minfo\_ml}} en tapant la commande : \commandBF{conda create -n minfo\_ml python=3.8}\\
  puis réponds aux questions pour télécharger et installer les paquets Python...
    \item Active l'EVP \DarkRed{\boldtt{minfo\_ml}} en tapant la commande : \commandBF{conda activate minfo\_ml}
\end{itemize}
Les modules nécessaires à l'APP sont ensuite installés en utilisant le fichier \file{requirements.txt} :\\[-6mm]
\begin{itemize}
  \item Positionne-toi dans le dossier \file{APP-ML} :\\
(Windows) $\leadsto$ \commandBF{cd C:\bsh<chemin du dossier APP-ML copié-collé avec le navigateur de fichiers>}\\
(Mac/Linux) $\leadsto$ \commandBF{cd /home/$\cdots$/APP-ML}
  \item Charge les modules en tapant la commande : \commandBF{pip install -r requirements.txt}\\
  Les modules et leurs dépendances sont téléchargés et installés...
  \item Complèter l'installation en tapant la commande :%
  \commandBF{conda install numpy pydot pydotplus}
\end{itemize}
\end{bclogo}  

%%%%%%%%%%%%%%%%%% diapo 4 %%%%%%%%%%%%%%%%%%%%%%%%%%%%%%%%%%%%%%%%%%%%%%%%%%%%%%%%%%%%%
\newpage
\vspace*{-16mm}
\section*{\ding{220} Séance 1 -- Auto-formation au {\em Machine Learning}}

\begin {bclogo}[noborder=true, couleurBarre=Chocolate, logo=\bctrombone]{}
  \vspace*{-4mm}\textbf{Auto-formation: travail personnel avec les {\em notebooks} de l'APP}\\[-2mm]

Les {\em notebooks} de l'APP permettent d'acquérir les savoir-faire nécessaires à la construction
de réseaux de neurones dense et convolutif, et à les entraîner à classifier les images de la banque MNIST (chiffres écrits à la main).

Chaque étudiant doit s'auto-former en travaillant les {\em notebooks} avec un processus {\em jupyter notebook} lancé dans l'EVP \DarkRed{\boldtt{minfo\_ml}}.

Les 3 premiers {\em notebooks} doivent être travaillés dans cette séance de travaille et le {\em notebook} \file{ML4\_CNN.ipynb} devrait être largement entamé...
\end{bclogo}  

%%%%%%%%%%%%%%%%% diapo 5 %%%%%%%%%%%%%%%%%%%%%%%%%%%%%%%%%%%%%%%%%%%%%%%%%%%%%%%%%%%%%%
\newpage
\vspace*{-16mm}
\section*{\ding{220} Séance 2 -- Fin de l'auto-formation, début du projet}

\begin {bclogo}[noborder=true, couleurBarre=Chocolate, logo=\bctrombone]{}
  \vspace*{-4mm}\textbf{Fin de l'auto-formation personnelle avec les {\em notebooks}}
  
  Fin du travail personnel avec le {\em notebook} \file{ML4\_CNN.ipynb}.
\end{bclogo}  

\begin {bclogo}[noborder=true, couleurBarre=Chocolate, logo=\bctrombone]{}
  \vspace*{-4mm}\textbf{Projet : Entraîner un réseau de neurone avec une banque de données spécifique}\\[-2mm]

Pour ce projet d'équipe, les étapes sont :
\begin{enumerate}
\item Choix d'une \textbf{banque de données spécifique} à votre projet à trouver sur Internet (images ou autre...).
\item Choix du réseau (dense ou convolutif) à entraîner, en utilisant les acquis d'apprentissage de votre auto-formation.
\item Entraînement supervisé du réseau de neurones avec la banque de données choisie, évaluation avec des données de test des performances du réseau entraîné.
\item Si possible, performance du réseau à classifier de nouvelles données (hors banque de données de test).  
\end{enumerate}

Le projet est conduit par l'équipe qui doit répartir entre ses membres les tâches à réaliser pour le projet.
\end{bclogo}  

%%%%%%%%%%%%%%%%%%%%%%%% diapo 6 %%%%%%%%%%%%%%%%%%%%%%%%%%%%%%%%%%%%%%%%%%%%%%%%%%%%%%%
\newpage
\vspace*{-16mm}
\section*{\ding{220} Séance 3 : projet et rendu}
%
\vspace*{-2mm}
\begin {bclogo}[noborder=true, couleurBarre=Chocolate, logo=\bctrombone]{}
  \vspace*{-4mm}\textbf{Projet : Entraîner un réseau de neurone avec une banque de données spécifique}
 
Suite et fin du projet.
\end{bclogo}
%
\vspace*{-5mm}
\begin {bclogo}[noborder=true, couleurBarre=Chocolate, logo=\bctrombone]{}
  \vspace*{-4mm}\textbf{Préparation du notebook de rendu du projet}

L'équipe prépare un rendu sous la forme d'un {\em notebook} qui pourra être créé par "copié/collé/modifié" de cellules des {\em notebooks} travaillés. 
Le fichier \file{template.ipynb} donne des exemples de mise en forme en utilisant des cellules {\em Markdown}.

Le {\em notebook} rendu devra à minima présenter les points suivants :
\begin{itemize}
\item Structure et contenu de la banque de données choisie.
\item Structure et entraînement du réseau de neurones choisi : stratégie d'entraînement et d'évaluation, résultats (courbes de performance...).
\item Si possible, performance du réseau à classifier de nouvelles données (hors banque de données d'entraînement ou de test).
\end{itemize}
\end{bclogo}  
%
\vspace*{-2mm}
En fonction de l'état sanitaire à la date de la dernière séance, le {\em notebook} de l'équipe sera présenté à l'oral à l'enseignant
sur une durée de 20 minutes dans la dernière heure de la dernière séance, ou bien envoyé à l'enseignant au plus tard une semaine après la dernière séance.

\vspace*{-2mm}
\bigskip
Critères d'évaluation du rendu: \\
- aptitude à mettre en oeuvre les acquises d'apprentissage visés;\\
- aptitude à présenter le contenu du projet;\\
- aptitude à présenter et critiquer les résultats obtenus.\\



\end{document}


%%%%%%%%%%%%%%%%%%%%%%%%%%%%%%%%%%%%%%%%%%%%%%%%%%%%%%%%%%%%%%%%%%%%%%%%%%%%%%%
\newpage
\section*{\ding{220} Séance 3 : fin du projet et rendu}

\ding{220} \textbf{Fin du projet, préparation du rendu}

L'équipe prépare un rendu sous la forme de 9 diapositives au maximum présentant les principales étapes du projet et les résultats obtenus à chaque étape.\\

Le rendu est présenté à l'enseignant chargé d'évaluer l'équipe pendant la dernière demi-heure de la séance 3.

\bigskip
Critères d'évaluation : \\
- aptitude à mettre en oeuvre les compétences acquises;\\
- aptitude à présenter le contenu du projet ;\\
- aptitude à critiquer les résultats obtenus ;\\
- aptitude individuelle à répondre aux questions.\\
\end{document}
