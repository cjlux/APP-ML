\documentclass[12pt,serif,mathserif,compress]{beamer}
\usepackage{pgf}
\usepackage{pgfpages}
\usepackage[T1]{fontenc}
\usepackage[utf8]{inputenc}

\usepackage{comment}
\usepackage{geometry}
\usepackage[most]{tcolorbox}
\tcbuselibrary{skins}
\usepackage{beamerthemesplit}
\usepackage{amsmath, amsfonts, epsfig, xspace}
\usepackage{pstricks,pst-node}
\usepackage{multimedia}
\usepackage{wasysym}

\usepackage{graphicx}% for including figures
\usepackage{tikz}
\usetikzlibrary{positioning,decorations.pathreplacing,arrows}

%%%%%%%%%%%%%%%%%%  Couleurs %%%%%%%%%%%%%%%%%%%%%%%%%%%
\definecolor{mauve}{rgb}{0.5,0,0.7}
\definecolor{carmin}{rgb}{0.7,0,0}
\definecolor{bleu}{rgb}{0,0,0.7}
\definecolor{marron}{rgb}{0.6,0.35,0}
\definecolor{vert}{rgb}{0,0.5,0}
\definecolor{gris}{rgb}{0.9,0.9,0.9}
\definecolor{no}{rgb}{1,0.9,1}
%%%%%%%%%%%%%%%%%%%%%%%%%%%%%%%%%%%%%%%%%%%%%%%%%%%%%%%%%%

\definecolor{White}        {rgb}{1.0,1.0,1.0}

\definecolor{VeryDarkBlue} {RGB}{0,0,153}
\definecolor{DarkBlue}     {rgb}{0.0,0.0,0.6}
\definecolor{Blue}         {rgb}{0.0,0.0,1.0}
\definecolor{MidBlue}      {rgb}{0.6,0.6,1.0}
\definecolor{LightBlue}    {rgb}{0.8,0.8,1.0}
\definecolor{VeryLightBlue}{rgb}{0.9,0.9,1.0}

\definecolor{Gray}{rgb}{0.7,0.7,0.7}
\definecolor{LightGray}{rgb}{0.94,0.94,0.94}

\definecolor{DarkGreen}{rgb}{0,0.6,0}
\definecolor{MidGreen}{rgb}{0.6,1,0.6}
\definecolor{LightGreen}{rgb}{0.88,1,0.88}
\definecolor{VeryLightGreen}{rgb}{0.9,1,0.9}

\definecolor{Yellow}{rgb}{1,1,0.4}
\definecolor{MidYellow}{rgb}{1,1,0.5}
\definecolor{LightYellow}{rgb}{1,1,0.6}
\definecolor{VeryLightYellow}{rgb}{1,1,0.9}

\definecolor{DarkRed}{rgb}{0.7,0.,0.}
\definecolor{Red}{rgb}{0.8,0.,0.}
\definecolor{LightRed}{rgb}{1,0.8,0.8}
\definecolor{VeryLightRed}{rgb}{1,0.9,0.9}

\definecolor{Mauve}{rgb}{0.7,0.,0.7}

\definecolor{Magenta}{rgb}{1,0.,1}
\definecolor{LightMagenta}{rgb}{1,0.5,1}

\definecolor{Chocolate}{rgb}{0.54,0.2,.004}
\definecolor{DarkChocolate}{RGB}{103,38,.00}

\definecolor{DarkOrange}{rgb}{0.65,0.35,0.}
\definecolor{Orange}{rgb}{.9,0.4,0.}
\definecolor{LightOrange}{rgb}{1,0.8,0.5}

\definecolor{Brown}{rgb}{0.4,0.4,0.}

\definecolor{LightCyan}{rgb}{0.5,1.,1.}

\definecolor{PBG} {rgb}{1.0,1.0,0.6}
\definecolor{PTXT}{rgb}{0.2,0.2,0.0}
\definecolor{IBG} {rgb}{0.8,0.8,1.0}
\definecolor{ITXT}{rgb}{0.2,0.2,0.0}
\definecolor{OBG} {rgb}{0.8,1.0,0.8}
\definecolor{OTXT}{rgb}{0.0,0.4,0.0}

\newcommand{\CommentVeryLightBlue}[1]{{\textcolor{VeryLightBlue}{#1}}}
\newcommand{\CommentWhite}[1]{{\textcolor{White}{#1}}}

\newcommand{\VeryDarkBlue}[1]{\textcolor{DarkBlue}{#1}}
\newcommand{\DarkBlue}[1]{\textcolor{DarkBlue}{#1}}
\newcommand{\Blue}[1]{\textcolor{Blue}{#1}}
\newcommand{\LightBlue}[1]{\textcolor{LightBlue}{#1}}
\newcommand{\VeryLightBlue}[1]{\textcolor{VeryLightBlue}{#1}}

\newcommand{\DarkGreen}[1]{\textcolor{DarkGreen}{#1}}
\newcommand{\LightGreen}[1]{\textcolor{LightGreen}{#1}}
\newcommand{\VeryLightGreen}[1]{\textcolor{VeryLightGreen}{#1}}
\newcommand{\MidGreen}[1]{\textcolor{MidGreen}{#1}}

\newcommand{\DarkRed}[1]{\textcolor{DarkRed}{#1}}
\newcommand{\Red}[1]{\textcolor{red}{#1}}
\newcommand{\LightRed}[1]{\textcolor{LightRed}{#1}}
\newcommand{\VeryLightRed}[1]{\textcolor{VeryLightRed}{#1}}

\newcommand{\Gray}[1]{\textcolor{gray}{#1}}

\newcommand{\Black}[1]{\textcolor{black}{#1}}

\newcommand{\White}[1]{\textcolor{white}{#1}}

\newcommand{\Mauve}[1]{\textcolor{Mauve}{#1}}

\newcommand{\LightMagenta}[1]{\textcolor{LightMagenta}{#1}}
\newcommand{\Magenta}[1]{\textcolor{Magenta}{#1}}

\newcommand{\DarkOrange}[1]{\textcolor{DarkOrange}{#1}}
\newcommand{\Orange}[1]{\textcolor{Orange}{#1}}
\newcommand{\LightOrange}[1]{\textcolor{LightOrange}{#1}}

\newcommand{\DeepPurple}[1]{{\textcolor[rgb]{0.3,0.,0.6}{#1}}}

\newcommand{\Brown}[1]{{\textcolor{Brown}{#1}}}
\newcommand{\Chocolate}[1]{\textcolor{Chocolate}{#1}}
\newcommand{\DarkChocolate}[1]{\textcolor{DarkChocolate}{#1}}

\definecolor{DBluePy}{RGB}{ 28, 78, 99}
\definecolor{BluePy} {RGB}{ 60,110,131}
\definecolor{MBluePy}{RGB}{200,210,240}
\definecolor{LBluePy}{RGB}{220,230,240}
\definecolor{DOranPy}{RGB}{248,194,  3}
\definecolor{OranPy} {RGB}{255,220, 70}
\definecolor{GreenPy}{RGB}{237,255,204}

\newcommand{\bsh}{\textbackslash}
\newcommand{\bshbsh}{\textbackslash\textbackslash}
\newcommand{\chevrons}{\ttfamily>\hspace*{-0.25mm}>\hspace*{-0.25mm}>\hspace*{0.28mm}\xspace}
\newcommand{\M}[1]{\Mauve{#1}}
\newcommand{\DG}[1]{\DarkGreen{#1}}
\newcommand{\DR}[1]{\DarkRed{#1}}
\newcommand{\B}[1]{\Blue{#1}}
\newcommand{\BPy}[1]{\BluePy{#1}}
\newcommand{\VDB}[1]{\VeryDarkBlue{#1}}
\newcommand{\DO}[1]{\DarkOrange{#1}}
\newcommand{\Or}[1]{\Orange{#1}}
\newcommand{\Choco}[1]{\Chocolate{#1}}
\newcommand*{\truc}{\Gray{\textbullet}}

\newcommand{\bif}[1]{{\ttfamily \M{#1}}}	% Python built in function
\newcommand{\typ}[1]{{\ttfamily \M{#1}}}  % Python built in type
\newcommand{\key}[1]{{\ttfamily \Or{#1}}}
\newcommand{\str}[1]{{\ttfamily \DG{#1}}}
\newcommand{\com}[1]{{\ttfamily \DR{#1}}}
\newcommand{\out}[1]{{\ttfamily \B{#1}}}
\newcommand{\command}[1]{{\ttfamily \Choco{#1}}}
\newcommand{\code}[1]{{\ttfamily \Choco{#1}}}
\newcommand{\file}[1]{{\ttfamily \VDB{#1}}}

\newcommand{\bifBF}[1]{\textbf{\bif{#1}}}	% Python built in function
\newcommand{\typBF}[1]{\textbf{\typ{#1}}}  % Python built in type
\newcommand{\keyBF}[1]{\textbf{\key{#1}}}
\newcommand{\strBF}[1]{\textbf{\str{#1}}}
\newcommand{\comBF}[1]{\textbf{\com{#1}}}
\newcommand{\outBF}[1]{\textbf{\out{#1}}}
\newcommand{\commandBF}[1]{\textbf{\command{#1}}}
\newcommand{\codeBF}[1]{\textbf{\code{#1}}}
\newcommand{\fileBF}[1]{\textbf{\file{#1}}}

\newcommand{\bfchoco}[1]{\textbf{\Chocolate{#1}}}
\newcommand{\bfdarkchoco}[1]{\textbf{\DarkChocolate{#1}}}


\usetheme{jlcKeynote}
\useoutertheme[subsection=false]{miniframes}
\setbeamercolor{background canvas}{bg=gray!50!white}
\setbeamercolor{structure}{bg=white, fg=gray}
\setbeamertemplate{itemize items}{\gray{$\CIRCLE$}}

\title[\hspace*{.5\linewidth}\insertframenumber/\inserttotalframenumber]
      {\fontsize{17}{17}\selectfont{Understanding \& using \\[2mm] \textbf{Deep Reinforcement Learning (DRL)}}}
\subtitle{Season 2 -- DQN algorithm}
\author[JLC -- {\tiny{nov 2018}} \hfill]{{Jean-Luc.Charles\,@\,ENSAM.EU}\\[1mm]\includegraphics[height=1.1cm]{images/Logo_AMPT_Bordeaux-2.png}}
\institute{}
\date{}
\titlegraphic{\vspace*{-1.6cm}\includegraphics[height=3.cm]{images/robot.png}}

\logo{}
\tcbset{enhanced, boxrule=0.2pt, sharp corners, drop lifted shadow, colback=Chocolate!25!white,colframe=Chocolate!75!black, fonttitle=\large}

\renewcommand\ttdefault{lmtt}

\begin{document}

\frame[plain]{\titlepage}

\setbeamercolor{structure}{fg=gray!50!white}

\section{Reinforcement Learning}

%===============================================================================
\begin{frame}{Machine Learning in Artificial intelligence}
  \includegraphics[width=\textwidth]{images/AI-from_MachineLearningForHumans-DRL.png}\\
  {\centering\tiny (figure from \href{https://medium.com/machine-learning-for-humans/why-machine-learning-matters-6164faf1df12}
                  {medium.com/machine-learning-for-humans/why-machine-learning-matters-6164faf1df12})}
\end{frame}
%===============================================================================

\subsection*{Why}

%===============================================================================
\begin{frame}{Why learn RL ?}
  \begin{itemize}
  \item <1-> Not just for games
  \item <1-> Make optimal decisions
  \item <1-> Solve problems
  \item <1-> Control/command algorithms
  \item <1-> It's novel
  \item <1-> Has a large potential for advancing AI
  \end{itemize}
\end{frame}
%===============================================================================

%===============================================================================
\begin{frame}{Where can we use RL ?}
  \begin{itemize}
  \item <1-> Robotics 
  \item <1-> Self-driving cars
  \item <1-> Inventory management
  \item <1-> Financial investments
  \item <1-> Medicinal diagnostic
  \item <1-> Decision-based situations
  \item <1-> Natural language processing
  \item <1-> Computer vision
  \item <1-> Internet of things
  \item <1-> ...
  \end{itemize}
\end{frame}
%===============================================================================

\subsection{RL main ingredients}

%===============================================================================
\begin{frame}
  \begin{tcolorbox}[title=RL main ingredients,fonttitle=\Large]
    
    \includegraphics[width=.9\textwidth]{images/RL-2.png}
    
      \bigskip
      \centering\bfdarkchoco{Agent -- action -- Environment -- state -- reward}
  \end{tcolorbox}  
\end{frame}
%===============================================================================

%===============================================================================
\begin{frame}
  \begin{tcolorbox}[title=RL:  \textbf{Agent} -- Environnement]
    The \bfdarkchoco{Agent} is the \bfdarkchoco{algorithm}:
    \begin{itemize}
    \item <2-> Monitors the \bfdarkchoco{Environment}
    \item <3-> Decides wich \bfdarkchoco{action} to be taken
    \item <4-> Action can be\\
      \bfdarkchoco{discrete}: on/off, left/right...\\
      \bfdarkchoco{continuous}: force/velocity applied....
    \end{itemize}
  \end{tcolorbox}
  \bigskip
  \visible<5->{\Chocolate{Dicrete} versus \Chocolate{continuous} action involves very different algorithms for the learning stage}
\end{frame}
%===============================================================================

%===============================================================================
\begin{frame}
  \begin{tcolorbox}[title=RL: Agent -- \textbf{Environnement}]
    The \bfdarkchoco{Environment} is what the Agent wants to monitor:    
    \begin{itemize}
    \item <2-> receives \bfdarkchoco{actions} from the Agent
    \item <3-> takes a new \bfdarkchoco{state} under the Agent's action
    \item <4-> gives back its new \bfdarkchoco{state} and a \bfdarkchoco{reward} to the Agent
    \item <5-> modelized as a \bfdarkchoco{Partially Observable Markov Process}
    \end{itemize}    
  \end{tcolorbox}
\end{frame}
%===============================================================================

\section{Neural Networks}

%===============================================================================
\begin{frame}{RL with Artificial Neural Networks}

  \subsection*{}

  Many computing technics can be used to implement the Machine Learning algorithms:
    \begin{itemize}
    \item <1-> \textbf{genetic programming}
    \item <1-> \textbf{Bayesian inference}, Fuzzy logic
    \item <1-> \textbf{Artificial Neural Networks}
    \item <1-> ...
    \end{itemize}    

    \bigskip
    \visible <2->{We will focus on \bfdarkchoco{Artificial Neural Networks}.}
\end{frame}
%===============================================================================

\subsection*{Artificial neuron principle}

%===============================================================================
\begin{frame}
  
  \tikzset{%
  neuron/.style={
    circle,
    draw,
    minimum size=1cm,
    font=\large
  },
  squa/.style={
    draw,
    inner sep=2pt,
    font=\large,
    join = by -latex
  },
  }
  \begin{tcolorbox}[title=The Artificial neuron model]  
    \hspace*{-.7cm}
    \begin{tikzpicture}[x=1.4cm, y=1.cm]

      \node [label=above:\parbox{2cm}{\centering Input\\stimuli}] at (0, 1.5) (x1)  {$x_1$};
      \node [] at (0, 0.5) (x2) {$x_2$};
      \node [] at (0, -0) (vdots) {$\vdots$};
      \node [] at (0, -0.7) (xn) {$x_n$};
      \node [label=above:\parbox{2cm}{\centering Bias}] at (2, 2) (bias) {$b$};
      \node [label=above:\parbox{2cm}{\centering Output}] at (4, 0.15) (y) {$y = f(\sum_i{w_{i}\,x_i} - b)$};
      
      \node [neuron/.try] (output) at (2,0.15) {\large{$\displaystyle{\Sigma | f}$}};
      
      \draw [o-latex] (x1) -- (output);
      \draw [o-latex] (x2) -- (output);
      \draw [o-latex] (xn) -- (output);
      \draw [o-latex] (bias) -- (output);
      \draw [->] (output) -- (y);

      \node [] at (1,1) () {$w_1$} ;
      \node [] at (1,.5) () {$w_2$} ;
      \node [] at (1, -0.45) () {$w_n$} ;
    \end{tikzpicture}
  \end{tcolorbox}
  \smallskip
  \visible<2->{An \bfdarkchoco{artificial neuron}:
    \begin{itemize}
    \item <3-> receives the input stimuli $(x_{i})_{i=1..n}$ with \textbf{weights} $(w_i)$
    \item <4-> computes the \textbf{weighted sum} of the input $\sum_i{w_{i}\,x_i}$
    \item <5-> outputs its \textbf{activation} $f(\sum_i{w_{i}\,x_i} - b)$, with $f$ a non-linear function.
    \end{itemize}
  }

\end{frame}
%===============================================================================

\subsection{Activation functions}

%===============================================================================
\begin{frame}{Activation functions}

  TODO \\
  ideas:\\
    threshold activation function\\
    unit step activation function\\
    sigmoid activation function - differentiable \\
    linear activation function \\
    Gaussian activation function \\
    relu activation function


\end{frame}
%===============================================================================

\subsection{Artificial Neural Networks}

%===============================================================================
\begin{frame}{Neural Networks}

  Feed-Forward Network (Single Layer  or Multiple Layer)
  Reccurent Network
  
\end{frame}
%===============================================================================

%===============================================================================
\begin{frame}{Neural Networks Architecture}
  
  
\end{frame}
%===============================================================================

\subsection{NN learning}

%===============================================================================
\begin{frame}{Neural Networks learning}
  
\end{frame}
%===============================================================================

\section{OpenAI}

%===============================================================================
\begin{frame}{\\[-4mm] \includegraphics[width=.2\textwidth]{images/openai-homepage.png} }
  \begin{tcolorbox}[title=Who is OpenAI ? What do they do ?]
    non profit artificial intelligence research company :
    \setbeamertemplate{itemize items}{\Chocolate{$\CIRCLE$}}
    \begin{itemize}
    \item build safe AGI (Artificial General Intelligence)
    \item ensure AGI's benefits are as widely and evenly distributed as possible
    \item Create open-source software
    \item Decision-based situations
    \end{itemize}
  \end{tcolorbox}  
  \setbeamertemplate{itemize items}{\gray{$\CIRCLE$}}
\end{frame}
%===============================================================================

\section{The CartPole problem}

\subsection{The CartPole}
%===============================================================================
\begin{frame}{The CartPole problem}
  
\end{frame}
%===============================================================================

\subsection{Solving CartPole with OpenAI Gym DQN}
%===============================================================================
\begin{frame}{Solving CartPole with OpenAI Gym DQN}
  
\end{frame}
%===============================================================================

\subsection{V-REP simulation}

%===============================================================================
\begin{frame}{V-REP}
  
\end{frame}
%===============================================================================

%===============================================================================
\begin{frame}{V-REP CartPole Simulation}
  
\end{frame}
%===============================================================================

%===============================================================================
\begin{frame}{V-REP Python API}
  
\end{frame}
%===============================================================================

%===============================================================================
\begin{frame}{OpenAI Gym with V-REP CartPole simulation}
  
\end{frame}
%===============================================================================

\section{The real CartPole}

\subsection{CartPole banc}

%===============================================================================
\begin{frame}{CartPole banc}
  
\end{frame}
%===============================================================================

\subsection{CartPole banc driving with Arduino}

%===============================================================================
\begin{frame}{CartPole banc driving with Arduino}
  
\end{frame}
%===============================================================================

%===============================================================================
\begin{frame}{CartPole banc driving with Arduino \& DQN}
  
\end{frame}
%===============================================================================



\end{document}

\begin{frame}
  \only<1>{Salut c'est only, je suis présent qu'au premier slide.\\}
  \visible<2->{Salut c'est visible, je suis visible à partir du slide 2.\\}
  \uncover<3->{Salut c'est uncover, je suis découvert à partir du slide 3.\\}
  \invisible<2-4>{Salut c'est invisible, je serais invisible du slide 2 au slide 4.\\}
  \alt<2>{Salut, je suis le alt qui sera au slide 2.\\}{Salut je suis le alt qui sera aux autres slides que la 2.\\}
  \temporal<2-3>{Salut je suis le temporal visible du slide 1}{Et moi le temporal visible du slide 2 au slide 3}{Et moi le temporal visible après le slide 3}
\end{frame}

%\begin{itemize}
%\item Reinforcement Learning (RL) is a branch of Machine Learning \visible<2->{a branch of Artificial Intelligence}
%\item <3-> RL involves an \textbf{agent} and an \textbf{environement}
%\item <4-> The \textbf{agent} learns optimal actions for maximizing \textbf{reward} given by the \textbf{environement}
%\end{itemize}

Le cerveau humain possède 100 milliards de neurones avec 20 000 synapses par neurone....


==========================================================================================================
\begin{frame}{Artificial neuron principle}
  
  \tikzset{%
  neuron/.style={
    circle,
    draw,
    minimum size=1cm,
    font=\large
  },
  squa/.style={
    draw,
    inner sep=2pt,
    font=\large,
    join = by -latex
  },
}
\hspace*{-.7cm}
\begin{tikzpicture}[x=1.4cm, y=1.cm]

  \node [label=above:\parbox{2cm}{\centering Input\\stimuli}] at (0, 1.5) (x1)  {$x_1$};
  \node [] at (0, 0.5) (x2) {$x_2$};
  \node [] at (0, -0) (vdots) {$\vdots$};
  \node [] at (0, -0.7) (xn) {$x_n$};
  \node [label=above:\parbox{2cm}{\centering Bias}] at (2, 2) (bias) {$b$};
  \node [squa, label=above:{\parbox{2cm}{\centering Activation\\function}}] at (4.1, 0.15) (F) {$f$};
  \node [label=above:\parbox{2cm}{\centering Output}] at (5.8, 0.15) (y) {$y = f(\sum_i{w_{i}\,x_i} - b)$};
  
  \node [neuron/.try] (output) at (2,0.15) {\Huge{$\displaystyle\Sigma$}};
  
  \draw [o-latex] (x1) -- (output);
  \draw [o-latex] (x2) -- (output);
  \draw [o-latex] (xn) -- (output);
  \draw [o-latex] (bias) -- (output);
  \draw [->] (output) -- (F);
  \draw [->] (F) -- (y);

  \node [] at (1,1) () {$w_1$} ;
  \node [] at (1,.5) () {$w_2$} ;
  \node [] at (1, -0.45) () {$w_n$} ;
  \node [] at (3, 0.3) () {\tiny{$\sum_i{w_{i}x_i} - b$}};
\end{tikzpicture}

\bigskip
\visible<2->{An \bfdarkchoco{artificial neuron}:
\begin{itemize}
\item <3-> receives the input stimuli $(x_{i})_{i=1..n}$ through \textbf{weights} $w_i$
\item <4-> computes the \textbf{weighted sum} of the input, minus a \textbf{bias} $b$
\item <5-> outputs its \textbf{activation} $f(\sum_i{w_{i}\,x_i} - b)$, with $f$ a non-linear function.
\end{itemize}
}

\end{frame}
=============================================================================================
